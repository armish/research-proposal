% This work by B. Arman Aksoy is licensed under a Creative Commons Attribution-ShareAlike 4.0 International License.

\documentclass[11pt,letterpaper]{article}
\usepackage[utf8]{inputenc}
\usepackage[english]{babel}
\usepackage{cite}
\usepackage[margin=1in]{geometry}
\usepackage{paralist}

\title{Research Proposal}
\author{Bulent Arman Aksoy}
\date{}

\begin{document}

\maketitle

% Make sense => decoding, deciphering, mining? for more inspiration?
\section*{Making sense of cancer data: Implications for personalized therapy and cancer biology}
\paragraph{Overview:}
My research interests lie at the intersection of computational and experimental work with a focus on Systems Biology, Genomics and Cancer Biology.
I am specifically interested in interpreting cancer data to discover therapeutic opportunities and novel biological mechanisms that can be experimentally investigated.
My research plans include
\begin{inparaenum}[(i)]
 \item developing computational methods to analyze cancer data for generating biological hypotheses (dry laboratory);
 \item using cell lines as models, testing these hypotheses \textit{in vitro} with the help of basic biological techniques, such as genome editing and high-throughput profiling (wet laboratory).
\end{inparaenum}

\paragraph{Motivation:}
In the very near future, all cancer patients coming into the clinics will have their genomic material profiled,
and we will need computational approaches that can make sense out of these data sets to enable more effective cancer therapies.
Drawing on what we learned from large-scale projects like The Cancer Genome Atlas (TCGA), 
new clinical trials that recruit patients based on their genomic profiles for specific targeted therapies are already being established;
but, there are still two major challenges in translating what we learn from these data sets into the clinic:
\begin{inparaenum}[(i)]
 \item a majority of the alterations that we see in patients are not immediately actionable;
 \item targeted therapies are short-lived, so resistance to such therapies eventually arise and the tumor progresses as a result.
\end{inparaenum}

\paragraph{Past work:}
Making use of public knowledge bases and cancer genomics data, 
I previously showed that 
\begin{inparaenum}[(i)]
 \item random passenger genomic events can create patient-specific therapeutic vulnerabilities that can be exploited by targeted drugs \cite{aksoy2014prediction};
 \item comprehensive analysis of cancer genomics data sets can reveal interesting biological insights about specific alteration events \cite{aksoy2014cancer}
\end{inparaenum}.

Moreover, I have helped develop computational tools that let other researchers better 
investigate cancer genomics data (cBioPortal \cite{gao2013integrative, cerami2012cbio}),
integrate biological pathway data (Pathway Commons and BioPAX \cite{demir2013using,babur2014integrating})
into their analysis and query available therapeutic targeted drugs (PiHelper \cite{aksoy2013pihelper}).
Finally, I have helped others to integrate these data sources into their own computational approaches \cite{ciriello2013emerging,korkut2014perturbation,babur2014systematic}.

\paragraph{Future work:} 
My first goal is to experimentally validate previous computational predictions on therapeutic vulnerabilities in wet laboratory
and for this, establish an experimental work flow that can efficiently screen cell line models for these types of vulnerabilities.
My second goal is to work on developing innovative cancer therapy strategies that turn the biology of cancer cells against themselves and prevent emergence of resistance in tumors to the targeted therapies.

\subsection*{Research goals}
\paragraph{Identification and testing of individualized therapeutic vulnerabilities in cancer:}
A majority of alterations observed in cancer samples occur at low frequency across patient cohorts
and do not confer cells any proliferative advantage--they are therefore called passenger alterations.
I previously showed that a considerable number of passenger alterations create therapeutic vulnerabilities that can be exploited with the use of a targeted drug, creating a unique personalized medicine opportunity for cancer patients \cite{aksoy2014prediction}.
By making use of data from cancer cell line screening projects,
I showed that a majority of the vulnerabilities identified in cancer patients also exist in these cell lines,
enabling testing of these predictions in wet laboratory using cell lines as models.

As the next step, I will extend the computational approach to analyze more data sets (both tumor and cell line data) and also to cover all known cellular pathways for identifying vulnerabilities of therapeutic interest.
I then will test the most common vulnerabilities that are seen in patients by introducing clinically relevant drugs to the matching cell line models that share to the matching cell line models that share the vulnerability of interest with patients.
In parallel, I will also utilize genome editing tools (CRISPR/Cas9 in particular)
to introduce genomic events of interest into cell line models to screen for pairs of genes that are predicted to be synthetic lethal for the cell.
The computational-prediction part of this work will significantly reduce the number of genes that are of interest, 
and therefore will make this approach more feasible.

\paragraph{Systematic characterization of drug resistance mechanisms in tumors to improve cancer therapy strategies:}
One of the hallmarks of the cancer cells is their genomic instability, 
which means that cancer cells are more likely to acquire new mutations compared to normal ones.
This enables cancer cells accumulate alterations that confer advantage to them under any selected condition.
This is also one of the reasons why tumors respond to targeted drugs for a relatively short time,
where resistance to targeted therapy eventually emerges due to evolutionary dynamics of the cancer cells
and the targeted therapy fails for most patients.

The importance of this evolutionary characteristic on understanding of cancer biology is twofold.
First, analysis of large-scale genomics projects lets us see recurring patterns in tumors from different patients, which helps reveal finer details of cancer biology.
I previously showed that comprehensive analysis of cancer data could show us previously unrecognized properties of molecular mechanisms -- \textit{e.g.} microRNA processing \cite{aksoy2014cancer}.
Second, the evolutionary perspective clarifies how cancer cells often respond to external perturbation by acquiring new mutations that counteract the inhibitory effect.

I propose that the latter response can be exploited for therapeutic purposes,
by using a carefully planned therapy with multiple targeted drugs to induce desired genomic alterations into the cancer cells.
The idea is to inhibit specific cellular targets until the cell acquires a set of new alterations that will make it intrinsically unstable and likely to undergo apoptosis or senescence.

This approach requires an extensive understanding of resistance mechanisms to be able to exploit them for therapeutic purposes.
I propose to systematically study these mechanisms in cell lines for drugs that are of high clinical interest.
Specifically, I will first introduce mutations into cell lines that will significantly lower the fidelity of DNA replication,
causing them to accumulate mutations relatively faster.
I will next propagate these cell lines for a determined time to promote genetic diversity within this cell population
and introduce the drug of interest to the environment to select for cells that are resistant to the particular drug.
Finally, I will characterize the genomes of resistant cells with the help of pooled sequencing.

This evolutionary and pooled method will accelerate uncovering common resistance mechanisms
and will enable to computationally identify them in a fast and efficient manner.

\clearpage

\footnotesize{
\bibliography{proposal}{}
\bibliographystyle{plain}}
\end{document}
