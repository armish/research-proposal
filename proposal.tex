\documentclass[12pt,letterpaper]{article}
\usepackage[utf8]{inputenc}

\title{Research Proposal}
\author{B. Arman Aksoy}

\renewcommand{\thesection}{}
\renewcommand{\thesubsection}{}


\begin{document}

\maketitle

\section{Making sense of cancer data: Implications on personalized therapy and basic biology}
Cancer is a complex disease.
Every cancer is unique in terms of genomic alterations it has.
Aggregated across all patients, some events are frequent and some are not.
Most studies focus on common, shared and frequent events.
These are important in the context of old-fashioned, cancer-type-specific clinical trials
because the target population is huge and big pharma can be persuaded to come up with a drug.

But infrequent, yet recurrent events are interesting, too.
Most rare events can be combined by functional grouping for boosting power.
People usually do predictions on these events,
yet no follow-up experimental work.
These events can tell us interesting stories about human biology
and can sometimes be exploited for therapy reasons.

Studies like TCGA, CCLE and CGP showed that massive profiling data is immensely important.
Many such projects are follow up and they are going to get bigger and bigger.
Frequent events will always be frequent as the data set gets bigger and bigger.
And there are many studies already focusing on these events.
Considering there is not going to be a single cure for cancer,
these relatively infrequent events are of high therapeutic interest.
With the emerging basket clinical trial concept,
they will become even more interesting.

Have been doing computational work ad, but originally trained as a molecular biologist.
Want to have a wet-lab component for my lab.
cell lines will be models for cancer.
FDA-approved or clinical trial drugs are going to be of interest.

\subsection{Aim 1: Identification and testing of therapeutic individualized vulnerabilities in cancer}
\subsection{Aim 2: Exploring cancer-specific genomic alterations that reveal biological facts}
\subsection{Aim 3: Exploiting evolutionary dynamics of cancer cells for driving them to more vulnerable states}

\end{document}
